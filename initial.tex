\documentclass{article}

\usepackage[]{float}
\usepackage[utf8]{inputenc}
\usepackage[]{hyperref}
\usepackage[hyperref=true,
    natbib=true,
    style=numeric,
    backend=bibtex]{biblatex}
\bibliography{references.bib}

\title{LS Project Proposal: TCP Congestion Algorithms In Datacenters}
\date{November 16, 2017}
\author{Luc Gommans, Rick van Gorp}

\begin{document}
\maketitle

\section{Introduction}

Multitentant datacenters aim to provide services with high availability and
reliability. Some datacenters experience the issue that parts of their networks
are sometimes congested, which might result in lower availability. Congestion
occurs when a buffer of the receiving network device is full and the network
device drops packets. TCP has a mechanism to avoid congestion in the network:
TCP congestion control. Multiple algorithms are available for congestion
control and show a different behaviour depending on the network's
characteristics. In 2010 algorithm DCTCP was described in an article
\cite{dctcp-congestion-original} and published as RFC8257 in 2017
\cite{dctcp-congestion}. DCTCP is optimized for datacenters and provides a
high-burst tolerance, low latency, and high throughput when the datacenter has
a small part of the buffer available \cite{dctcp-congestion}. In 2016 one
algorithm was proposed: BBR. BBR is a TCP congestion algorithm created by
Google that achieves higher bandwidths and lower latencies compared to other
TCP congestion methods \cite{bbr-congestion}. A comparison of BBR with CUBIC
\cite{bbr-congestion-comparison} shows that the BBR node pushes the CUBIC node
away in bandwidth when using small buffers. The BBR node gets more bandwidth
allocated than the CUBIC node.

We propose to do a performance and behaviour analysis of different TCP
congestion control algorithms in a datacenter network. Based on the results we
will recommend datacenters on measures to keep providing an available and
reliable environment for customers when multiple congestion control algorithms
active in the network.


\section{Research Questions}

Our main research question is: How can a datacenter provide an available and
reliable environment for customers when multiple TCP congestion control
algorithms are used in the network?

This main research question divides in five sub questions:

\begin{enumerate}
	\item What TCP congestion algorithms are mostly implemented in datacenter environments?
	\item How do the TCP congestion algorithms work?
	\item How do the TCP congestion algorithms differ in behavior?
	\item What is the impact on the availability and service for customers when using multiple TCP congestion algorithms?
	\item What connection management measures are able to control the connection when using different congestion control algorithms?
\end{enumerate}


\section{Scope}

During this research we will focus on at least the TCP congestion control
algorithms BBR, CUBIC, DCTCP. Additional congestion control algorithms will be
determined after researching what the most implemented TCP congestion
algorithms are in datacenter environments. We will conduct a behavior analysis
and comparison between those algorithms, including the impact of mixed usage of
the algorithms on the availability of a service in a datacenter. We will look
at the measures a datacenter has to take to keep their service available and
reliable for all customers when running multiple congestion algorithms at the
same time.


\section{Planning}

\begin{table}[H]
	\centering
	\label{my-label}
	\begin{tabular}{l|p{8cm}}
		\textbf{Week} & \textbf{Activity} \\ \hline
		1    & Conducting desk research on TCP congestion control algorithms. Conducting desk research on the congestion and packet loss occuring in different scenarios. Conducting desk research on bandwidth management in data centers. Setup the testing environment. \\
		2    & Finish test environment setup and conduct all tests.\\
		3    & Process the results and start writing the project report.        \\
		4    & Finalize our project report and process the final results. Setting up the presentation and practice it.       
	\end{tabular}
	\caption{Project planning}
\end{table}


\section{Ethical Considerations}

For this research we will create a private setup for testing the performance of
TCP congestion control algorithms. This research does only involve TCP
congestion control algorithms made available publicly. Therefore we do not
expect any ethical issues during this research.


\section{Related work}

The TCP congestion control algorithm BBR is discussed in \cite{bbr-congestion}.
The journal contains a performance test of the BBR algorithm and a comparison
between the BBR algorithm and the CUBIC algorithm, where a noticeable
difference in bandwidth allocation is shown when using small buffers. In
\cite{multiple-congestion} a performance analysis of multiple TCP congestion
control algorithms is performed. This research does not include the new
congestion control algorithms we want to analyze and does not include the
datacenter context. In \cite{dctcp-congestion-original} an algorithm is
proposed to avoid congestion within datacenters. The authors describe an
algorithm that maintains a small buffer occupancy and has an early detection of
congestion.

\printbibliography

\end{document}

